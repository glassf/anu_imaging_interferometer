\thispagestyle{empty}
Covering previous Warr\cite{WARR1998}\cite{WARR1997} and Oliver\cite{HOWARD2006}\cite{OLIVER2006} systems and motivation for this system.
Note new or unique aspects, particularly imaging (not wavefront detection).

Laser interferometry is a standard tool for measurement of the two-dimensional plasma electron density distribution in magnetically confined plasmas \cite{DONNE1995, HARTFUSS1997}.
The line-integrated electron number density is obtained from the phase shift imparted on the probing beam.
Beam access constraints have resulted in a variety of discrete-multi-channel and scanning interferometer configurations that attempt to maximize the number of available viewing chords.\cite{KAWAHATA1997,JIANG1995,ROMMERS1997,CANTON2006}.

The H-1 heliac \cite{HAMBERGER1990}, is a flexible, medium scale helical-axis stellarator located at the Australian National University.
The coil-in-tank construction allows excellent diagnostic access to the plasma cross section.
To take advantage of this, various multi-view scanning interferometers have been developed to allow tomographic reconstruction of the plasma density distribution \cite{WARR1997,HOWARD1992:2,HOWARD2006}.
Spatially scanning a single beam across the plasma maximizes the available power probing the plasma and reduces the system cost.
However, temporal resolution is limited by the achievable electronic or mechanical sweep periods to $\gsimeq 1$ ms.
In order to study high frequency density fluctuations ($<$ 100 kHz) in H-1, we have undertaken to construct a fast, multi-channel, millimeter-wave imaging interferometer.
